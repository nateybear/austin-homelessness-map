\documentclass[../main]{subfiles}

If one is inclined to believe the identifying assumptions I've presented, then they could reasonably conclude that these results constitute evidence of absence of an effect. That is, the sudden and unprecedented policy reversal by Austin City Council did not create any knock-on effects for crime, and in an indirect sense they might conclude that homelessness does not portend crime. I do not endorse this inference with such enthusiasm, and my inclination is more towards concluding there is an absence of evidence. Conditional on a richer set of covariates, this analysis may be more convincing. 

Ultimately, we must accept that this vein of analysis has limitations. For one, it is difficult to infer conclusions on the movements and encampment decisions of the unsheltered homeless based solely on citation data. These data do not include information on sweeps conducted by state law enforcement, and moreover, the idea of a ``treatment effect'' is fairly abstract in this case, as I previously discussed.

Second, it is hard to pin down what time-varying heterogeneity may be affecting the level of crime in the city. In general, inclusion of a fixed effect is sufficient to capture some of the classical determinants of crime, such as income and demographic composition, which are relatively static within a census tract. Then again, Austin is a rapidly changing city. There are demographic pressures from the influx of high-profile companies like Google, Facebook, Samsung, and Amazon. Austin is one of the fastest-growing cities in the United States, and new arrivals tend to be upper-income, white-collar workers. For any resident of Austin, it is obvious that this rapid gentrification may introduce time-varying heterogeneity in both income and demographics.

Moreover, I would argue that a structural model that makes a more genuine effort to enumerate the rational motivations behind both the encampment decisions of the unsheltered homeless and the larcenous decisions of a criminal actor would produce more convincing results. My assignment of treatment based on the upper decile of mean citations is admittedly na{\"i}ve and subject to the Lucas critique in the sense that it makes a rigid assumption about the reaction of the unsheltered homeless to the City Council's policy change. A more thorough approach would be to identify structural parameters that are invariant to policy changes, although developing and estimating a structural model of homelessness would certainly be a sizable and novel task.