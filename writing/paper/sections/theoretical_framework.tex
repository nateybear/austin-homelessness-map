\documentclass[../main.tex]{subfiles}

The main hurdle to overcome in answering causal questions about the unsheltered homeless is that there is little data on their whereabouts. The Ending Community Homelessness Coalition of Austin (ECHO) conducts a yearly point-in-time (PIT) count of all the unsheltered people living in the city. The irregularity of the sample makes it difficult to conduct meaningful inference, and the survey is subject to methodological issues (including, for instance, that the survey is conducted by volunteers and reliability is hence dependent on volunteer turnout).

Instead, I will try to conduct inference based on the anti-homeless citation data that I have. This means I will have to develop some stylized facts that describe the relationship between the issuance of citations and the location decisions of unsheltered people.

TODO: citations written is a function of policy and presence of homeless people
TODO: citations are only written in areas around downtown/urban areas

Generally, I will assume that a citation has a transitory deterrent impact. The PIT always finds that the same parts of Austin have the highest unsheltered counts, namely areas around downtown (see Figure \ref{fig:pit_maps}). If enforcement policy had any permanent effect then this would not be the case. Rather, it seems that the deterrent impact is fairly small given the rate of recidivism and missed court appearances. Hence, for this paper I will assume that the number of citations issued is a suitable proxy for the number of unsheltered people in an area.

TODO: discuss Lucas critique with response of homeless to city policy

TODO: WHAT EXACTLY IS THE TREATMENT EFFECT?

TODO: talk about a model of crime