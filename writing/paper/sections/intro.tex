\documentclass[../main.tex]{subfiles}
It is common in America for cities to issue various citations to unsheltered people in order to deter them from establishing residence in certain areas. In Austin, Texas, this has come in the form of city ordinances banning public camping, aggressive solicitation, and sitting or lying on public sidewalks (I generally refer to all three of these ordinances as ``the camping ban''). These ordinances have been a de facto norm in the city since the 1990s, and an analysis published by the \textit{Texas Observer} found that 20,000 citations had been issued between January 2000 and May 2020. Of those cited, 75\% failed to appear and were issued arrest warrants for a Class C Misdemeanor \cite{observer_article}. 

Homelessness advocates point out that these city ordinances force people to camp in more isolated areas of the city, further from access to supportive services. Moreover, they argue that a camping ban effectively criminalizes and perpetuates homelessness, with arrest warrants preventing the unhoused from obtaining a job or apartment.

Proponents of these city ordinances generally argue that the homeless are either unseemly, bad for business, or commit crimes. This paper is an attempt to produce reduced-form evidence of this last claim. I exploit a sudden reversal in policy by the Austin City Council in 2018 to estimate the treatment effect of the camping ban on crime using a differences-in-differences approach. I discuss the limitations of this approach and conclude with a suggestion of a more convincing model.