\documentclass[../main.tex]{subfiles}
It is common in America for cities to issue various citations to unsheltered people in order to deter them from establishing residence in certain areas. In Austin, this has come in the form of city ordinances banning public camping, aggressive solicitation, and sitting or lying on public sidewalks (I will refer to these collectively as a ``camping ban'' or ``anti-homeless ordinances''). These ordinances have been a de facto norm in the city since the 1990s, and an analysis published by the \textit{Texas Observer} found that 20,000 citations had been issued between January 2000 and May 2020. Of those cited, 75\% failed to appear and were issued arrest warrants for a Class C Misdemeanor \cite{observer_article}. 

Unsheltered advocates point out that these city ordinances force people to camp in more isolated areas of the city, further from access to supportive services. Moreover, they argue that a camping ban effectively criminalizes and perpetuates homelessness, with arrest warrants preventing the unhoused from obtaining a job or apartment.

Proponents of these city ordinances generally argue that the homeless are either unseemly, bad for business, or commit crimes. This paper is an attempt to generate testable hypotheses from these claims. I exploit a sudden reversal in policy by the Austin City Council in 2018 to generate quasi-experimental variation and estimate a difference-in-differenced treatment effect for crime in areas of the city most affected by the change in policy. This reduced-form model finds no evidence of an increase in thefts and robberies when Austin stopped enforcing anti-homelessness ordinances. I also discuss identification issues and make suggestions for a richer structural model.